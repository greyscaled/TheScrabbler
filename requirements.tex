\documentclass[12pt, oneside]{article}   	
\usepackage{geometry}                		
\geometry{letterpaper}            
\usepackage{graphicx}					
\usepackage{amsmath, amssymb}
\usepackage{enumerate}
%----------------------------------------------------------------------------------------------------------------------------------------
\title{SFWR 2XB3: Requirements Documentation}
\author{Group 33 \\ 
	     Curtis Milo \\
	     Greg Barkans \\
	     Collin Gillespie \\
	     \vspace{2cm} \\
	     Due: Friday March 27, 2015 \\
	     Prof: Reza Semavi \\
	     Lab Section: 01 
	     }
\date{}							
%----------------------------------------------------------------------------------------------------------------------------------------

\begin{document}
\maketitle
\newpage
%----------------------------------------------------------------------------------------------------------------------------------------

%----------------------------SECTION 1 INTRO ---------------------------------------------------------------------------------

\section{The Scrabbler - Preliminary Details}
\subsection{Problem Statement}
Scrabble is a popular board game in which players use letter tiles to make words and score points.  The objective of the game is to out-score your opponent.  The problems are: what words yield the highest score and how does one gain an intuition for this?  These problems are precisely what The Scrabbler is being designed to solve.

%--------------------------SECTION 2 DOMAIN ---------------------------------------------------------------------------------

\section{Domain}
\begin{enumerate}[1.]
	\item The Scrabbler is to be implemented as a web browser application.  More specifically, The 		Scrabbler is to be a single page application.  Thus users must have access to the internet 		and a modern browser.    
	\item The Scrabbler is to be mobile, tablet, laptop and desktop browser friendly, thus increasing 		the availability to the intended audience.
	\item Users are intended to range from experienced, competitive Scrabbler players to children 			looking to improve their vocabulary.
	\item Another potential audience is those looking to implement new solutions for Scrabble 			computer AI.
	\item Stakeholders: Group 33 and Dr. Reza Samavi are the sole stakeholders.  
\end{enumerate}


%--------------------------SECTION 3 FUNCTIONAL REQUIREMENTS -------------------------------------------------

\section{Functional Requirements}

\begin{enumerate}[1.]
	\item The application will sequentially accept input from the user.  Specifically:
	\begin{itemize}
		\item The user will input letters representing the current state of a Scrabble game.
		\item The user will indicate which squares are special scoring squares, by selecting from a list
		\item The user will select a series of squares either horizontally or vertically representing 				where they would like to form a word.  The series may be at minimum 1 square, and at 			maximum 7.
		\item The user will input what letter tiles they have.
	\end{itemize}
	
	\item In order to control the flow of requirement 1., the application will have the following states:
	\begin{itemize}
		\item Accepting input from keyboard/keypad of existing letters
		\item Accepting input of special scoring tiles from a drop down list
		\item Waiting for a selection area on the grid
		\item Prompting the user with a text box that shall take as input the user's tiles
		\item Displaying the result
	\end{itemize}
	\item Each state may be succeeded by pressing ``enter", or the previous state may be returned to 		by pressing ``back"
	\item The app shall abide by Scrabble rules, specifically:
	\begin{itemize}
		\item Every sequence of adjacent letters forms a valid Scrabble word
		\item Scoring is the combined result of tile values and special board squares
	\end{itemize}
	\item Inputting letters in \textit{occupied squares} will replace the old letter with the new one
	\item The app shall return the highest-scoring valid Scrabble word for the selected area.  The word 		will not violate Scrabble rules in that it will form valid words with all adjacent letters. 
\end{enumerate}


%-------------------------SECTION 4 NON - FUNCTIONAL REQUIREMENTS ---------------------------------------

\section{Non-Functional \& System Requirements}
%------------4.1 SYSTEM REQs
\subsection{System Requirements}
\begin{enumerate}[1.] 
	\item The app shall not request any personal information in order to maximize web security
	\item The app shall behave reliably relative to internet connectivity.  Its function is not safety-critical 		and thus tolerances to internet fluctuations are allowable.
	\item The app shall perform as accurately as possible given the current Scrabble dictionary dataset (available: http://introcs.cs.princeton.edu/java/data/ospd.txt)
	\item The app shall avoid lengthy-stalls and perform within a reasonable timeframe.  Algorithmic 		optimizations are to be considered.
\end{enumerate}

%----------4.2 Constraints
\subsection{Constraints}
\begin{enumerate}[1.]
	\item The user will require internet connectivity
	\item The app will require a modern browser, which supports CSS3 and HTML5
	\item The app will not choose the best position for the user's word, rather the user selects where 		the app forms the best word.	
\end{enumerate}


%----------4.3 Interface
\subsection{User Interface}
\begin{enumerate}[1.]
	\item The app will display a 7 $\times$ 7 grid
	\item The user will click on cells of the grid via touchscreen or mouse
	\item The user will enter values into the cells of the grid via keyboard or touchscreen keypad
	\item The cells of the grid that are to represent special scoring tiles are first to be clicked via 			mouse or touchscreen.  Then a value from a drop down list is to be selected via a mouse 			click or touchscreen.
\end{enumerate}

%--------------------------SECTION 5 VERIFICATION MAINTENANCE ------------------------------------------------
\section{Verification \& Maintenance Requirements}
%---------5.1 Verification and testing
\subsection{Verification \& Testing}
\begin{enumerate}
	\item Verification steps to be made:
	\begin{itemize}
		\item Inputting canonical sequences in the interface, ensuring every event will result in the 			proper state
		\item Module verification will be made via tabular expressions/function tables
		\item Bottom-up testing will be implemented whereby smaller units are unit-tested before 
			larger methods that use these units
		\item (see 5.2 Maintenance) Because the design process is to be an agile/spiral process, 				changes are likely to occur.  Specification and Design documentation will be continually 			updated.  Each set of changes will be reviewed with the stakeholders.
	\end{itemize}
\end{enumerate}

%------5.2 Maintenance
\subsection{Maintenance}
\begin{enumerate}
	\item The Development team will use standard version control procedures.  Each member will 			maintain a separate patch.  The master copy will contain merged pull requests with every 			member of the design team agreeing on the merges.
	\item Git version control is to be used
	\item The project will be maintained at the location: \\
	https://github.com/TheHouseHippo/WebDev/tree/master/TheScrabbler
	\item An AGILE/spiralling design methodology is to be used, whereby functioning prototypes
		and proofs of concept will be quickly refactored and built upon, checking with the 				stakeholders at each milestone.
\end{enumerate}

%------5.3 Platform etc. Requirements
\subsection{Utility \& Other Requirements}
\begin{enumerate}[1.]
	\item The application will be mainly written in Javascript, allowing for any electronic device with an 		internet connection and a modern browser to access the application.  
	\item HTML5, CSS3 (Bootsrap) will be used to design a mobile and tablet friendly application.
	\item Scalability will be of major concern, such that the application will display correctly on screens 		of all sizes.
\end{enumerate}
\end{document}  
