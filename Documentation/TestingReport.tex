%--------------------------------TOP LEVEL SHIT-------------------------------------------------------------------------------
\documentclass[11pt, oneside]{report}   
\usepackage{geometry}                		
\geometry{letterpaper}                   		
\usepackage{graphicx}    
\usepackage{amssymb}
\usepackage{enumerate}
%--------------------------------------------------------------------------------------------------------------------------------------
\title{Sfwr Eng 2XB3 - Final Project}    
\author{ 
	Group 33  - The Scrabbler \\ \\         
	Greg Barkans \\
	Colin Gillespie \\
	Curtis Milos \\
	 \vspace{1cm} \\
	Prof: Dr. Samavi \\
	Lab: L01 \\
	Due: Friday April 10, 2015 \\
	\vspace{1cm} \\
}
\date{}		 			
%--------------------------------------------------------------------------------------------------------------------------------------
\begin{document}    
\maketitle                  

\textbf{NOTE:} An official Testing Report was NOT mentioned in the final project requirements, but since we had to perform testing anyways our group included this as proof.  We also did our entire implementation from scratch (including our own data structures etc) and didn't rely on libraries and thus testing our implementation is very critical. 
\newpage     
\tableofcontents   
\newpage             
\listoffigures         
\newpage
\listoftables
\newpage

%------------------------------------------------------SECTION 1-----------------------------------------------------------------

\section*{Preface}
This document is organized with each chapter dedicated to a different testing procedure type (from low-level unit testing to higher level system testing).  Within each chapter,  there is a section representing a conceptual model.  From there, there are subsections for each class module and sub subsections for each unit/method within those classes.  


%----------------------------------------------------------------------------------------------------------------------------------------
%------------------------------------------------------CHAPTER 1-----------------------------------------------------------------
%----------------------------------------------------------------------------------------------------------------------------------------

\chapter{Unit Testing}
%-- Intro paragraph
The manner in which each of these tests were done was systemically consist.  Specifically, each unit was tested on its own with necessary stubs and drivers.  This sort of testing is considered to be whitebox and will verify the method on its own, but will not imply integration/system testing (to be done in other ways).  This manner of testing will verify the \textsc{functional correctness} of the requirements for the particular method, and only that method.

%-----------------------Chapter 1, section 1: Model----%
\section{Model}

%-----hl Indicies
\begin{table}[H]
\caption{private method this.hlIndicies}
\begin{center}
\begin{tabular}{|p{0.25\textwidth}|p{0.3\textwidth}|p{0.3\textwidth}|p{0.05\textwidth}|}

\hline
\textbf{TEST CASE} & \textbf{EXPECTED RESULT} & \textbf{ACTUAL RESULT} & P/F \\
\hline
d: ``horizontal" x:3, y:4 with 3 highlighted cells and no letters & [0, 1, 2] & [0, 1, 2] & P \\
\hline
d: ``vertical" x:1, y:1 with 4 highlighted cells and a letter in the second one & [0,2,3] & [0,2,3] & P \\
\hline
d: ``horizontal" x:0, y:0 with 7 highlighted cell and a letter in the first and last cell & [1, 2, 3, 4, 5] & [1, 2 , 3 ,4 ,5] & P \\
\hline
\end{tabular}
\end{center}
\label{default}
\end{table}%

%-----match Tiles
\begin{table}[H]
\caption{public method this.matchTiles}
\begin{center}
\begin{tabular}{|p{0.25\textwidth}|p{0.3\textwidth}|p{0.3\textwidth}|p{0.05\textwidth}|}

\hline
\textbf{TEST CASE} & \textbf{EXPECTED RESULT} & \textbf{ACTUAL RESULT} & P/F \\
\hline
words in heap: side, dies, ides, peep. Horizontal highlight from [0][0] to [0][3]. Tiles: s,i,d,e
& [side, ides, dies] & [side, ides, dies] & P \\
\hline
words in heap: side, dies, ides, peep. Horizontal highlight from [0][0] to [0][3]. Tiles: none & [ ] & [ ] & P \\
\hline
words in heap: side, dies, ides, peep. Horizontal highlight from [0][0] to [0][3]. Tiles: i,d,e,wildcard & [side, ides, dies] & [sides, ides, dies] & P \\
\hline


\hline
\end{tabular}
\end{center}
\label{default}
\end{table}%

%----create Regex
\begin{table}[H]
\caption{public method this.matchTiles}
\begin{center}
\begin{tabular}{|p{0.25\textwidth}|p{0.3\textwidth}|p{0.3\textwidth}|p{0.05\textwidth}|}

\hline
\textbf{TEST CASE} & \textbf{EXPECTED RESULT} & \textbf{ACTUAL RESULT} & P/F \\
\hline
cells [0][0] to [0][3] highlighted, no letters & /\textbackslash w \textbackslash w \textbackslash w \textbackslash w/ & /\textbackslash w \textbackslash w \textbackslash w \textbackslash w/ & P \\
\hline
cells [1][2] to [4][2] highlighted, an a in cell [1][2] & / a \textbackslash w \textbackslash w \textbackslash w / & / a \textbackslash w \textbackslash w \textbackslash w / & P \\
\hline
cells[2][2] to [5][2] highlight, an a in cell [2][2], a b in cell [3][1] & / a [aeioy] \textbackslash w \textbackslash w / & / a [aeioy] \textbackslash w \textbackslash w / & P \\
\hline
\end{tabular}
\end{center}
\label{default}
\end{table}%

%-----------------------Chapter 1, section 2: View----%
\section{View}





%-----------------------Chapter 1, section 3: Controller----%
\section{Controller}


%----------------------------------------------------------------------------------------------------------------------------------------
%------------------------------------------------------CHAPTER 1-----------------------------------------------------------------
%----------------------------------------------------------------------------------------------------------------------------------------

\end{document}  